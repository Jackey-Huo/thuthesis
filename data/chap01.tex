% !TeX root = ../main.tex

\chapter{移动操作机器人的国内外现状}
\label{cha:intro}

\section{移动操作机器人概念}
\label{cha:base_comcept}

移动操作机器人(Mobile Manipulator)目前仍然是一个非常宽泛的概念,通常我们认为,典型的移动操作
机器人是由一台机械臂和一个可自由移动的平台组成的,机械臂通过某种方式挂载在平台上\cite{bostelman2016survey}。
近年来,随着科技的飞速发展以及制造业转型的巨大压力,移动操作机器人成为新的研究热点,在学术界及
工业界受到了广泛关注\cite{schneier2015literature}。移动操作机器人已经被广泛的应用到了航天\cite{ambrose2004mobile}、医药、制造\cite{guizzo2011meka}、清洁、服务等多个场景,
在各个领域发挥着重要的作用。但移动操作机器人在性能和功能上还远远没有达到我们期待的水平,相
关研究及工业尝试还需要很长时间的发展。

广义的移动操作机器人除机械臂外,可依赖的平台非常多样,除了常见的轮式底盘、平衡车式底盘外,还可
使用无人机作为移动平台,甚至仿人的双足行走底盘。无人机平台的移动操作机器人有夹爪式、机械臂式等
各种种类,近年来也广泛的
受到各界的关注\cite{ruggiero2018aerial}。相比于无人机或者足式底盘,轮式底盘的移动操作机器人负载更
高,性能更强劲,应用更加广泛,功能更加多样,品种也更加丰富,因此本文主要论述以轮式底盘作为移动
平台的移动操作机器人,并给出相应的控制方法。

\section{移动操作机器人国内外应用现状}
\label{cha:application}

近年来,国内外机器人相关的创业公司层出不穷,移动操作机器人行业也是投资机构聚焦的热点。下面本文
将分别介绍移动操作机器人的两大主要部分:机械臂以及移动底盘的应用历史与现状,随后介绍移动操作机器人
作为整体的应用情况。

机械臂作为主要生产工具在制造业承担作用的历史非常久,机械臂在制造业最早的规模化应用被认为诞生
在汽车制造领域。同其他行业相比,汽车领域产品款式固定,出货量大,规模效应强,这些特性使得基于硬编码
控制的大型高精度机械臂得到广泛应用,如图~\ref{fig:armcar} 展示了工业机械臂在汽车产线上应用
的场景。以汽车工业为代表的自动化机器人应用大部分为高精度工业机械臂,其发展已经有近
半个世纪的历史,自1973年ABB和KUKA将工业机器人推向市场后,这一行业迎来了飞速发展,工业机械臂
的厂商不断增加,款式与性能持续丰富,各种面对细分领域的工业机械臂相继面世,如图~\ref{fig:industry_arms}。

\begin{figure}
\centering
\begin{subfigure}{1.\textwidth}
  \centering
  \includegraphics[height=3.5cm]{arm1.jpg}
  \includegraphics[height=3.5cm]{arm2.png}
  \includegraphics[height=3.5cm]{arm4.jpg}\\
  \includegraphics[height=3.5cm]{arm3.jpg}
  \includegraphics[height=3.5cm]{arm5.jpg}
  \includegraphics[height=3.5cm]{arm6.jpeg}
\end{subfigure}%
\caption{各式工业机械臂产品}
\label{fig:industry_arms}
\end{figure}


目前,国际
工业机器人领域有四大标杆企业,分别是瑞典 ABB、德国 KUKA、日本FANUC 和日本安川电机。四大厂商
各有所长,ABB擅长控制系统,KUKA优势在于系统集成应用与本体制造,FANUC长于数控系统,安川电机
的优势在于伺服电机制造和运动控制器的研发。除四大头部企业外,美国 Adept Technology、瑞士
Staubli、意大利Comau、日本的川崎、爱普生、那智不二越和中国新松机器人自动化股份有限公司也是
国际工业机器人的重要供应商\cite{huangxihuanReview}。

除汽车工业外,高精度工业机械臂在医疗
领域也取得了不俗的成绩,Intuitive Surgical公司研发的“达芬奇”手术机器人在世界范围内广泛应用,
我国于2006年引入第一套达芬奇手术系统,至今各式手术机器人在中国已累计完成10万例外科手术,如
图~\ref{fig:davinci}。我国华大智造研发的远程
超声检测系统~\ref{fig:huada},该系统在新冠疫情期间的武汉方舱医院内进行了应用测试\cite{wushengzheng20205g},取得了可喜的进展。


\begin{figure}
\centering
\begin{subfigure}{.5\textwidth}
  \centering
  \includegraphics[height=3cm]{kuka_car.jpg}
  \caption{完全由库卡机械臂组成的汽车焊接流水线}
\end{subfigure}%
\begin{subfigure}{.5\textwidth}
  \centering
  \includegraphics[height=3cm]{abb_car.jpeg}
  \caption{ABB机械臂组成的汽车零件装配产线}
\end{subfigure}
\caption{工业机械臂在流水线上工作}
\label{fig:armcar}
\end{figure}

\begin{figure}
\centering
\begin{subfigure}{.5\textwidth}
  \centering
  \includegraphics[height=4cm]{davinci.jpeg}
  \caption{达芬奇机器人在工作}
  \label{fig:davinci}
\end{subfigure}%
\begin{subfigure}{.5\textwidth}
  \centering
  \includegraphics[height=4cm]{huada.png}
  \caption{华大智造的远程超声检测机器人}
  \label{fig:huada}
\end{subfigure}%
\caption{机械臂产品在国内外医疗领域的应用}
\end{figure}



\iffalse % 删掉这些华而不实的玩意儿
    近年来,一些富有展示性的机械臂应用开始逐步面世,例如机器人调酒应用。
    最早引起轰动的是皇家加勒比游轮有限公司在其的最新款邮轮海洋量子号上的“仿生酒吧”中应用了机器人
    调酒师,如图~\ref{fig:liangzi_arm},很快各种调酒机器人被竞相开发出来,如意大利的机器人
    酒吧\ref{fig:italy_arm},拉斯维加斯的机器人调酒师\ref{fig:las_arm},以及我国哈工大开发的
    机器人调酒师\ref{fig:hgd_arm}。尽管机器人调酒目前并不能带来成本或者质量
    上的提升,但是机器人调酒的概念本身富有科技感,新鲜有趣充满噱头,是吸引消费者的一大亮点。


    \begin{figure}
    \centering
    \begin{subfigure}{.5\textwidth}
      \centering
      \includegraphics[height=3.5cm]{liangzi_arm.jpg}
      \caption{皇家加勒比海洋量子号的仿生酒吧}
      \label{fig:liangzi_arm}
    \end{subfigure}%
    \begin{subfigure}{.5\textwidth}
      \centering
      \includegraphics[height=3.5cm]{italy_arm.jpg}
      \caption{意大利The View机器人酒吧}
      \label{fig:italy_arm}
    \end{subfigure}%
    \\
    \begin{subfigure}{.5\textwidth}
      \centering
      \includegraphics[height=3.5cm]{las_arm.jpg}
      \caption{拉斯维加斯微醺机器人酒吧}
      \label{fig:las_arm}
    \end{subfigure}%
    \begin{subfigure}{.5\textwidth}
      \centering
      \includegraphics[height=3.5cm]{hgd_arm.jpg}
      \caption{哈工大开发的机器人调酒师}
      \label{fig:hgd_arm}
    \end{subfigure}%
    \caption{机械臂调酒师}
    \end{figure}

\fi

上述这些产品注重控制精度与末端负载,功耗巨大,重量也非常可观,几乎没有移动能力。
尽管工业机器人行业至今已发展多年,体量巨大,技术积累丰厚,但企业的产品研发主要集中在
工业机械臂的硬件性能方面,如提升精度与负载,提升安全性等等。近年来随着电商、物流行业
的迅猛发展,基于机械臂的智能分捡领域开始
蓬勃发展,国内涌现出一批专注物品分捡、分类的创业公司,如深圳蓝胖子机器人~\ref{fig:lanpangzi}、
北京梅卡曼德机器人~\ref{fig:mechmind}等,这些企业专注于算法或者算法硬件集成,致力于
机器人的自动化与智能化。


\begin{figure}
\centering
\begin{subfigure}{.5\textwidth}
  \centering
  \includegraphics[height=4cm]{lanpangzi_robot.jpeg}
  \caption{蓝胖子机器人研发的机械臂分捡系统}
  \label{fig:lanpangzi}
\end{subfigure}%
\begin{subfigure}{.5\textwidth}
  \centering
  \includegraphics[height=4cm]{mechmind.jpg}
  \caption{梅卡曼德研发的机械臂操作系统在工作}
  \label{fig:mechmind}
\end{subfigure}
\end{figure}

除了专注于操作的工业机械臂,专注移动导航的自动导航车(AGV,Automated Guided Vehicles)
在制造业、运输业也大放异彩。AGV产品最早出现在上世纪50年代,于70年代左右开始应用于
制造业\cite{黄志球2010自动导航车}。按
功能可将AGV分为自动搬运车、自动拖车、自动叉车等几类,或者按照导航方式可分为电磁引导、激光引导、
惯性引导等方式。目前AGV在汽车厂(如通用、丰田、大众等)的制造和装配线上都得到了广泛的使用。
相比于工业机械臂,AGV的机械、电子结构并不复杂,控制算法也相对成熟,制造难度不大。我国的海康威视
就是一个重要的AGV生产商,如图~\ref{fig:haikang}展示了海康威视研发的系列AGV产品。

\begin{figure}
\centering
\begin{subfigure}{.5\textwidth}
  \centering
  \includegraphics[width=.7\linewidth]{haikang1.png}
  \caption{移载AGV}
\end{subfigure}%
\begin{subfigure}{.5\textwidth}
  \centering
  \includegraphics[width=.7\linewidth]{haikang2.png}
  \caption{移载AGV}
\end{subfigure}%
\\
\begin{subfigure}{.5\textwidth}
  \centering
  \includegraphics[width=.7\linewidth]{haikang3.png}
  \caption{叉车}
\end{subfigure}%
\begin{subfigure}{.5\textwidth}
  \centering
  \includegraphics[width=.7\linewidth]{haikang4.png}
  \caption{潜伏系列}
\end{subfigure}%
\caption{海康威视发布的各式AGV产品}
\label{fig:haikang}
\end{figure}

AGV产品除了在制造业广泛应用外,近年来也开始在服务业逐步发展,例如今年来十分火爆的海底捞
机器人餐厅配备了智能AGV送餐员(图~\ref{fig:haidilao}),瑞幸咖啡尝试的无人配送小车
(图~\ref{fig:luckin}),
智行者开发的无人自动清扫车(图~\ref{fig:zhixingzhe}), Yogo Robot开发的写字楼包
裹配送机器人(图~\ref{fig:yogo}),极木科技开发的智能车辆搬动机器人(图~\ref{fig:jimu})
等等。这些产品专注细分领域的需求,有一定的智能处理能力。

\begin{figure}
\centering
\begin{subfigure}{.5\textwidth}
  \centering
  \includegraphics[height=3.5cm]{haidilao.jpg}
  \caption{海底捞送餐机器人}
  \label{fig:haidilao}
\end{subfigure}%
\begin{subfigure}{.5\textwidth}
  \centering
  \includegraphics[height=3.5cm]{luckin.jpg}
  \caption{瑞幸咖啡的无人配送车}
  \label{fig:luckin}
\end{subfigure}%
\\
\begin{subfigure}{.5\textwidth}
  \centering
  \includegraphics[height=3.5cm]{zhixingzhe.jpg}
  \caption{智行者开发的无人自动清扫车}
  \label{fig:zhixingzhe}
\end{subfigure}%
\begin{subfigure}{.5\textwidth}
  \centering
  \includegraphics[height=3.5cm]{yogo.jpg}
  \caption{Yogo Robot开发的写字楼包裹配送机器人}
  \label{fig:yogo}
\end{subfigure}%
\\
\begin{subfigure}{.5\textwidth}
  \centering
  \includegraphics[height=3.5cm]{jimu.jpg}
  \caption{极木科技开发的智能车辆搬动机器人}
  \label{fig:jimu}
\end{subfigure}%
\caption{AGV机器人在服务业的应用}
\end{figure}


移动操作机器人本质上是机械臂产品与AGV的集合,但目前来看其商业应用方向与前两者有极大的区别。
现阶段移动操作机器人领域还没有出现在商业上十分成功的厂商,但是国内外出现了一大批专注这一
领域的创业公司和项目,致力于将移动操作机器人应用到医疗、服务、制造等场景下。例如Diligent Robotics
研发的机器人护士Moxi(如图~\ref{fig:moxi})专注于医院场景下的配送、运输业务;欧
盟发起的旨在推动工业环境中机器人与人写作的”第二双手“(SecondHand)项目的一期项目成
果,AMRAR-6机器人(如图~\ref{fig:armar6})能够在工业环境下与工人协作完成物品的搬动、维修等任务。

\begin{figure}
\centering
\begin{subfigure}{.5\textwidth}
  \centering
  \includegraphics[height=4.2cm]{moxi.jpg}
  \caption{Moxi机器人护士}
  \label{fig:moxi}
\end{subfigure}%
\begin{subfigure}{.5\textwidth}
  \centering
  \includegraphics[height=4.2cm]{armar6.jpeg}
  \caption{ARMAR-6仿人机器人}
  \label{fig:armar6}
\end{subfigure}%
\caption{国内外创业公司研发的医疗移动操作机器人}
\end{figure}

除上述两例面向特定任务的移动机器人外,产业界还有一部分产品定位为“仿人”的具有社交属性的服务
机器人,例如日本Toyota公司面对人口老龄化现象推出的家庭照料机器人HSR(Human Support Robot)\ref{fig:hsr},
以及Softbank Robotic发布的Pepper机器人\ref{fig:pepper}。这些机器人大都行动笨拙,但是有双臂、
手、头等和人类身体结构对应的机械结构,并且有一定的交互能力。

\begin{figure}
\centering
\begin{subfigure}{.5\textwidth}
  \centering
  \includegraphics[height=5cm]{hsr.jpeg}
  \caption{Toyota Human Support Robot}
  \label{fig:hsr}
\end{subfigure}%
\begin{subfigure}{.5\textwidth}
  \centering
  \includegraphics[height=5cm]{pepper.jpeg}
  \caption{Softbank Robotics Pepper}
  \label{fig:pepper}
\end{subfigure}
\label{fig:hsr_pepper}
\caption{仿人机器人产品}
\end{figure}

除了上述商业应用比较明确的公司以及产品外,国际上大名顶顶的波士顿动力公司(Boston Dynamics)也
研究发布了几款轮式移动操作机器人(如图~\ref{fig:bd_wheel}),和以大狗为移动平台挂载轻量
级机械臂的机器人(如图~\ref{fig:bd_dog})。这些机器人在运动控制上均达到了极高的水平,为
移动操作机器人的运动性能评价设定了新的标杆。

\begin{figure}
\centering
\begin{subfigure}{.5\textwidth}
  \centering
  \includegraphics[width=.6\linewidth]{bd_dog.jpg}
  \caption{SPOTMINI + Arm大狗机器人}
  \label{fig:bd_dog}
\end{subfigure}%
\begin{subfigure}{.5\textwidth}
  \centering
  \includegraphics[width=.6\linewidth]{bd_wheel.jpg}
  \caption{仓储搬运机器人Handle}
  \label{fig:bd_wheel}
\end{subfigure}
\caption{波士顿动力发布的移动操作机器人产品}
\end{figure}


\section{移动操作机器人国内外研究现状}
\label{cha:research}

机器人领域中学术界的研究总是走在工业界前面,移动操作机器人方向也不例外。最早的工业
机械臂于1969年由Victor Scheinman发明,称为“斯坦福机械臂”,随后Victor Scheinman在
MIT AI Lab设计了被成为“MIT arm”的机械臂并在一些公司的支持下开发了Puma机器人\cite{huangxihuanReview},
机器人最初诞生自高校的实验室里。在50年后的今天,学术界的研究依然指引着机器人发展的
方向。随着计算机技术的不断发展,深度学习以及各类控制算法的成熟,研究者们开始将目光
聚焦于提升机器人的“智能性”以及它的泛化性上。

学术领域在探索各式感知、控制、交互方法的同时,也自研或者培育了许多移动操作机器人产品。
这些产品机器人不但对领域内前沿的设计思想进行实践、对前沿的传感器进行尝试,方便了广大
研究人员,有些甚至取得了商业上的成功。例如在开源机器人领域著名的PR2机器人
(如图~\ref{fig:pr2})以及TurtleBot(如图~\ref{fig:turtlebot})机器人平台。

\begin{figure}
\centering
\begin{subfigure}{.6\textwidth}
  \centering
  \includegraphics[height=4cm]{pr2.jpeg}
  \caption{PR2服务机器人}
  \label{fig:pr2}
\end{subfigure}%
\begin{subfigure}{.4\textwidth}
  \centering
  \includegraphics[height=4cm]{turtlebot3.jpg}
  \caption{turtlebot3机器人}
  \label{fig:turtlebot}
\end{subfigure}
\caption{学术机器人平台}
\end{figure}

为了统一机器人算法研究的编程平台,降低方法的复现难度,促进社区的健康发展。广大机器人
研究学者们也参与推动维护了多个开源的机器人编程框架,其中包括应用广泛的ROS
(Robot Operating System)\cite{quigley2009ros},YARP
(Yet another robot platform)\cite{metta2006yarp}等框架。这些开源框架为社区提供了
极大的发展便利,同时某些面向学术研究的移动机器人操作平台也通过捆绑支持这些机器人
编程框架取得了商业上的成功,例如前文提到的PR2和TurtleBot机器人;Universal Robotics
研发的URx系列机械臂(图~\ref{fig:urx}); 以及国内的autolabor四轮
差速车(图~\ref{fig:autolaborpro1} ~\ref{fig:autolabor})等等产品。


\begin{figure}
\centering
\begin{subfigure}{.6\textwidth}
  \centering
  \includegraphics[height=3.5cm]{ur_series.jpg}
  \caption{Universal Robotics研发的系列UR机械臂}
  \label{fig:urx}
\end{subfigure}%
\begin{subfigure}{.4\textwidth}
  \centering
  \includegraphics[height=3.5cm]{autolaborpro1.jpg}
  \caption{Autolabor 大负载四轮差速车}
  \label{fig:autolaborpro1}
\end{subfigure}
\begin{subfigure}{.5\textwidth}
  \centering
  \includegraphics[width=.6\linewidth]{autolabor.jpg}
  \caption{Autolabor小型四轮差速平台}
  \label{fig:autolabor}
\end{subfigure}
\caption{支持ROS平台的机器人产品}
\end{figure}


机器人算法一直是科研领域的研究热点,按照功能可以将机器人相关的算法分为感知和控制两部分。
感知方向包括物品的检测、分类、定位等功能,得益于深度神经网络的快速发展,机器人的感知能力
近年来有了飞速的提高,借助经典的视觉检测方法配合机械臂规划方法实现的快速自动分捡方案
在各大比赛中不断出现。借助深度学习(如图~\ref{fig:img_seg})或者传统视觉方法、点云
处理方法(如图~\ref{fig:pc_seg})实现的拾取平面检测,抓取
点寻找,夹爪位姿寻找算法也不断涌现,并且取得了非常好的效果。

\begin{figure}
\centering
\begin{subfigure}{.5\textwidth}
  \centering
  \includegraphics[height=4cm]{img_seg.jpg}
  \caption{基于图片的物品分割识别}
  \label{fig:img_seg}
\end{subfigure}%
\begin{subfigure}{.5\textwidth}
  \centering
  \includegraphics[height=4cm]{pc_seg.jpeg}
  \caption{基于点云的物品分割效果}
  \label{fig:pc_seg}
\end{subfigure}
\caption{学术机器人平台}
\end{figure}


在导航定位方面,SLAM技术( simultaneous localization and mapping,即时定位与地图构建)
日趋成熟,基于视觉、点云或者多传感器融合的算法不断出现,视觉方面,直接法、特征点法
或者半直接的算法大量增加,特别的基于优化的方案有了长足的发展,被认为在未来的SLAM领域能够取代
滤波算法,如基于特征点法的ORB-SLAM\cite{mur2015orb},视觉与IMU融合的VINS\cite{qin2018vins},
半直接法的SVO\cite{forster2014svo},稀疏直接法的DSO\cite{engel2017direct} 等等,
如图~\ref{fig:slams};基于雷达的各类
SLAM算法进展也十分迅速,如基于粒子滤波的Gmapping\cite{grisettiyz2005improving},基于优化
的loam\cite{zhang2014loam},基于优化以及子图的Cartographer\cite{hess2016real}等等。激光
SLAM主要以产出2D地图和位姿为主,其算法结果可视化效果大多很相似,如图~\ref{fig:lidar_slams}

\begin{figure}
\centering
\begin{subfigure}{.6\textwidth}
  \centering
  \includegraphics[height=3.5cm]{orb.jpg}
  \caption{ORB-SLAM}
\end{subfigure}%
\begin{subfigure}{.4\textwidth}
  \centering
  \includegraphics[height=3.5cm]{vins.jpeg}
  \caption{VINS-Mono}
\end{subfigure}
\\
\begin{subfigure}{.59\textwidth}
  \centering
  \includegraphics[height=3.2cm]{svo.jpg}
  \caption{SVO}
\end{subfigure}
\begin{subfigure}{.4\textwidth}
  \centering
  \includegraphics[height=3.2cm]{dso.jpeg}
  \caption{DSO}
\end{subfigure}
\caption{基于视觉的各种SLAM算法效果图}
\label{fig:slams}
\end{figure}


\begin{figure}
\centering
\begin{subfigure}{.5\textwidth}
  \centering
  \includegraphics[height=3.7cm]{cartographer.jpg}
  \caption{雷达生成2D地图的可视化效果}
\end{subfigure}%
\begin{subfigure}{.5\textwidth}
  \centering
  \includegraphics[height=3.7cm]{cartographer3d.jpg}
  \caption{多线激光雷达生成的3D效果图}
\end{subfigure}
\caption{基于雷达的SLAM算法效果图}
\label{fig:lidar_slams}
\end{figure}


格式SLAM算法不断提高机器人主动定位的精度与鲁帮性的同时,导航算法的进步主要体现
在控制框架的创新和完善上,经过数十年的努力,机器人社区在导航方面已经形成了较为稳定的一套
方法,即基于Costmap的分层地图维护方法\cite{lu2014layered}和分别负责路径生成与控制命令
生成的双规划器算法\cite{guimaraes2016ros}。目前
在静态环境中的灵活避障导航问题已经基本被解决\cite{Zhou2017A}。 此外增强学习在机器人控制
上的应用也成为近年来算法方向的研究热点\cite{schaal2002learning}。

机械臂控制算法是机器人算法领域的一大方向,由于现代机械臂自由度高,冗余程度大,状态空间大,
对机械臂的规划算法提出了严峻挑战。学术界不断提出新的机械臂规划算法例如基于梯度优化的机械臂
路径寻找算法 CHOMP\cite{ratliff2009chomp},随机路径优化算法 STOMP\cite{kalakrishnan2011stomp}
等等。同时,一些维护机械臂相关算法 的开源库也迅速兴起,比如维护随机采样规划器的OMPL
(The Open Motion Planning Library)\cite{sucan2010open},
基于搜索算法的SBPL(The Search Based Planning Library)\cite{likhachev2014sbpl},以及
用于碰撞检测的FCL(The Fast Collision Library)\cite{pan2012fcl}。随着这些支撑组件的不断
成熟,一些集运动学解算、路径生成、碰撞检测功能于一体的一站式的机械臂规划框架也开始出现,
如基于ROS的MoveIt!\cite{chitta2012moveit}, V-REP等等,如图~\ref{fig:moveit_vrep}

\begin{figure}
\centering
\begin{subfigure}{.5\textwidth}
  \centering
  \includegraphics[height=4cm]{moveit.png}
  \caption{MoveIt!工作时的可视化窗口}
\end{subfigure}%
\begin{subfigure}{.5\textwidth}
  \centering
  \includegraphics[height=4cm]{vrep.jpeg}
  \caption{V-REP可视化窗口}
\end{subfigure}
\caption{开源的机械臂规划框架}
\label{fig:moveit_vrep}
\end{figure}


为促进研究成果在实际场景中的应用,各大公司会议等组织举办了许多世界级的面向特定场景的
机器人大赛,其中著名的有Amazon Picking Challenge\cite{wurman2016amazon},
RoboCup@Home\cite{wisspeintner2009robocup},IROS Manipulation Competition\cite{moon2017iros}。
这些比赛极大的促进了机器人社区的交流,在比赛期间也涌现出了许多优秀的机器人设计以及
方法。下面以RoboCup@Home为例详细的介绍机器人赛事如何组织及促进学术的发展交流。

RoboCup@Home比赛已经培育起一个完善的社区,每年RoboCup@Home的比赛任务
即由社区相关人员协助商议决定,RoboCup@Home组委会共同维护一份RuleBook\cite{rulebook}
,组委会成员
使用GitHub完成RuleBook的编写任务,该仓库是完全开放且欢迎参赛者提交修改与贡献的。
RoboCup@Home组委会中,设有专门的任务制定部门Technical Committee(TC),这部分成员
大都有深厚的相关从业背景和多年的研究经验,另外有一部分成员来自各个主力参赛队伍的
核心开发人员(比如本文作者作为Tinker机器人的主要程序员参与了2020年的TC工作),以
确保比赛规则同时兼顾学科研究热点和可完成性。

随着机器人行业相关技术的不断发展,RuleBook中的任务也在每年更新,并且向着越来越复杂、
越来越贴近真实生活场景的方向演进。在多年的比赛中,组委会在任务内容、任务形式上也作出
了很多改进与尝试,2019年的RoboCup@Home比赛就及其大胆的将任务数量进行了爆炸式的扩容,
尝试了很多新的复杂的任务,其中有一些被证明并不成功,因此2020年的规则制作过程中又将
不合理的任务去除,将任务数量固定为Stage I 5个,Stage II 4个。 在RoboCup@Home赛场上
出现了大量设计新奇,功能多样的机器人(如图~\ref{fig:other_teams})。

\begin{figure}
    \centering
    \begin{minipage}{.45\linewidth}
            \begin{subfigure}[t]{.9\linewidth}
                \includegraphics[width=\textwidth]{homer.png}
                \caption{homer@ UniKoblenz}
            \end{subfigure}
    \end{minipage}
    \begin{minipage}{.45\linewidth}
        \begin{subfigure}[t]{.8\linewidth}
            \includegraphics[width=\textwidth]{pumas.jpg}
            \caption{Pumas}
        \end{subfigure} \\
        \begin{subfigure}[b]{.8\linewidth}
            \includegraphics[width=\textwidth]{catie.jpg}
            \caption{CATIE Robotics}
        \end{subfigure} 
    \end{minipage}
    \caption{RoboCup@Home中出现的机器人}
    \label{fig:other_teams}
\end{figure}

\section{本文内容简介}

本章系统的介绍了移动操作机器人的基本定义和国内外研究、应用现状。后续章节将对本文的
主要工作——Tinker移动操作机器人进行介绍,详细的给出Tinker机器人的系统搭建,定位
导航算法的实现与调试,机械臂控制及相应的视觉算法实现以及各部分的版本迭代经过。





