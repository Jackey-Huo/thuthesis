% !TeX root = ../main.tex

\chapter{移动操作机器人的国内外现状}
\label{cha:intro}

\section{移动操作机器人概念}
\label{cha:base_comcept}

移动操作机器人(Mobile Manipulator)目前仍然是一个非常宽泛的概念,通常我们认为,典型的移动操作
机器人是由一台机械臂和一个可自由移动的平台组成的,机械臂通过某种方式挂载在平台上\cite{bostelman2016survey}。
近年来,随着科技的飞速发展以及制造业转型的巨大压力,智能机器人方向成为新的研究热点,在学术界及
工业界受到了广泛关注\cite{schneier2015literature}。移动操作机器人已经被广泛的应用到了航天\cite{ambrose2004mobile}、医药、制造\cite{guizzo2011meka}、清洁、服务等多个场景,
在各个领域发挥着重要的作用。但移动操作机器人在性能和功能上还远远没有达到我们期待的水平,相
关研究及工业尝试还需要很长时间的发展。

广义的移动操作机器人除机械臂外,可依赖的平台非常多样,除了常见的轮式底盘、平衡车式底盘外,还可
使用无人机作为移动平台。无人机平台的移动操作机器人有夹爪式、机械臂式等各种种类,近年来也广泛的
受到各界的关注\cite{ruggiero2018aerial}。相比于无人机底盘,轮式底盘的移动操作机器人负载更
高,性能更强劲,应用更加广泛,功能更加多样,品种也更加丰富,因此本文主要论述以轮式底盘作为移动
平台的移动操作机器人,并给出相应的控制方法。

\section{移动操作机器人国内外应用现状}
\label{cha:application}

近几年来,国内外机器人相关的创业公司层出不穷,智能机器人行业也是投资人聚焦的热点。机器人
作为主要生产工具在制造业承担作用的历史非常久,机器人在制造业最早的规模化应用被认为诞生
在汽车制造领域。同其他行业相比,汽车领域产品款式固定,出货量大,成本高,这些特性使得基于硬编码
控制的大型高精度机械臂得到广泛应用,如图\ref{fig:arm_car}展示了工业机械臂在汽车产线上应用
的场景。以汽车工业为代表的自动化机器人应用大部分为高精度工业机械臂,工业机器人的发展已经有近
半个世纪的历史,自1973年ABB和KUKA将工业机器人推向市场后,这一行业迎来了飞速发展。目前,国际
工业机器人领域有四大标杆企业,分别是瑞典 ABB、德国 KUKA、日本 FANUC 和日本安川电机,四大厂商
各有所长,ABB擅长控制系统,KUKA优势在于系统集成应用与本体制造,FANUC长于数控系统,安川电机
的优势在于伺服电机制造和运动控制器的研发。除四大头部企业外,美国 Adept Technology、瑞士
Staubli、意大利Comau、日本的川崎、爱普生、那智不二越和中国新松机器人自动化股份有限公司也是
国际工业机器人的重要供应商\cite{xihuanhuangReview}。除汽车工业外,高精度工业机械臂在医疗
领域也取得了不俗的成绩,c

\begin{figure}
\centering
\begin{subfigure}{.5\textwidth}
  \centering
  \includegraphics[width=.6\linewidth]{kuka_car.jpg}
  \caption{完全由库卡机械臂组成的汽车焊接流水线}
\end{subfigure}%
\begin{subfigure}{.5\textwidth}
  \centering
  \includegraphics[width=.6\linewidth]{abb_car.jpeg}
  \caption{ABB机械臂组成的汽车零件装配产线}
\end{subfigure}
\label{fig:arm_car}
\end{figure}

这些产品注重控制精度与末端负载,功耗巨大,重量也非常可观,几乎没有移动能力。
尽管目前的工业机器人行业至今以发展多年,体量巨大,技术积累丰厚,但企业的产品研发主要集中在
工业机械臂的硬件性能方面。近年来随着电商、物流行业的迅猛发展,基于机械臂的智能分捡领域开始
蓬勃发展,国内涌现出一批专注物品分捡、分类的创业公司,如深圳蓝胖子机器人\ref{fig:lanpangzi}、
北京梅卡曼德机器人\ref{fig:mechmind}等一批专注于算法或者算法硬件集成的企业。


\begin{figure}
\centering
\begin{subfigure}{.5\textwidth}
  \centering
  \includegraphics[width=.6\linewidth]{lanpangzi_robot.jpeg}
  \caption{蓝胖子机器人研发的机械臂分捡系统}
  \label{fig:lanpangzi}
\end{subfigure}%
\begin{subfigure}{.5\textwidth}
  \centering
  \includegraphics[width=.6\linewidth]{mechmind.jpg}
  \caption{梅卡曼德研发的机械臂操作系统在工作}
  \label{fig:mechmind}
\end{subfigure}
\label{fig:arm_car}
\end{figure}

除了专注于操作的工业机械臂,专注移动的产品自动导航车(AGV,Automated Guided Vehicles)
在制造业、运输业也大放异彩。AGV产品出现在上世纪50年代,于70年代左右开始应用于制造业。按
功能可将AGV分为自动搬运车、自动拖车、自动叉车等几类,或者按照导航方式可分为电磁引导、激光引导、
惯性引导等方式。目前AGV在汽车厂(如通用、丰田、大众等)的制造和装配线上都得到了广泛的使用。
相比于工业机械臂,AGV的机械、电子结构并不复杂,控制算法也相对成熟,制造难度不大。我国的海康威视
就是一个重要的AGV生产商,如图\ref{fig:haikang}展示了海康威视研发的系列AGV产品。

\begin{figure}
\centering
\begin{subfigure}{.4\textwidth}
  \centering
  \includegraphics[width=.6\linewidth]{haikang1.png}
\end{subfigure}%
\begin{subfigure}{.4\textwidth}
  \centering
  \includegraphics[width=.6\linewidth]{haikang2.png}
\end{subfigure}%
\begin{subfigure}{.4\textwidth}
  \centering
  \includegraphics[width=.6\linewidth]{haikang3.png}
\end{subfigure}%
\begin{subfigure}{.4\textwidth}
  \centering
  \includegraphics[width=.6\linewidth]{haikang4.png}
\end{subfigure}%
\label{fig:haikang}
\end{figure}

移动操作机器人本质上是机械臂产品与AGV的集合,但目前来看其商业应用方向与前两者有极大的区别。
现阶段移动操作机器人领域还没有出现在商业上十分成功的厂商,但是国内外出现了一大批专注这一
领域的创业公司,致力于将移动操作机器人应用到医疗、服务制造等场景下。例如Diligent Robotics
研发的机器人护士Moxi\ref{fig:moxi}专注与医院场景下的配送、运输业务;我国华大智造研发的远程
超声检测系统\ref{fig:huada},该系统在新冠疫情期间武汉方舱医院内进行了应用测试\ref{wushengzheng20205g},
取得了可喜的进展。

\begin{figure}
\centering
\begin{subfigure}{.5\textwidth}
  \centering
  \includegraphics[width=.6\linewidth]{moxi.jpg}
  \label{fig:moxi}
\end{subfigure}%
\begin{subfigure}{.5\textwidth}
  \centering
  \includegraphics[width=.6\linewidth]{huada.png}
  \label{fig:huada}
\end{subfigure}%
\label{fig:haikang}
\end{figure}

除上述两例面向特定任务的移动机器人外,产业界还有一部分产品定位为“仿人”的具有社交属性的服务
机器人,例如日本Toyota公司面对人口老龄化现象推出的家庭照料机器人HSR(Human Support Robot)\ref{fig:hsr},
以及Softbank Robotic发布的Pepper机器人\ref{fig:pepper}。这些机器人大都行动笨拙,但是有双臂、
手、头等和人类身体结构对应的机械结构,并且有一定的交互能力。

\begin{figure}
\centering
\begin{subfigure}{.5\textwidth}
  \centering
  \includegraphics[width=.6\linewidth]{hsr.jpeg}
  \caption{Toyota Human Support Robot}
  \label{fig:hsr}
\end{subfigure}%
\begin{subfigure}{.5\textwidth}
  \centering
  \includegraphics[width=.6\linewidth]{pepper.jpeg}
  \caption{Softbank Robotics Pepper}
  \label{fig:pepper}
\end{subfigure}
\label{fig:hsr_pepper}
\end{figure}

除了上述商业应用比较明确的公司以及产品外,国际上大名顶顶的波士顿动力公司(Boston Dynamics)也
研究发布了几款轮式移动操作机器人\ref{fig:bd_wheel}或者以大狗为移动平台挂载轻量级机械臂的机器人\ref{fig:bd_dog}。

\begin{figure}
\centering
\begin{subfigure}{.5\textwidth}
  \centering
  \includegraphics[width=.6\linewidth]{bd_dog.jpg}
  \caption{SPOTMINI + Arm大狗机器人}
  \label{fig:bd_dog}
\end{subfigure}%
\begin{subfigure}{.5\textwidth}
  \centering
  \includegraphics[width=.6\linewidth]{bd_wheel.jpg}
  \caption{仓储搬运机器人Handle}
  \label{fig:bd_wheel}
\end{subfigure}
\end{figure}


\section{移动操作机器人国内外研究现状}
\label{cha:research}


\section{Tinker机器人的设计目标}

清华大学未来机器人团队成立于2012年,旨在帮助对机器人领域各个方面感兴趣的
同学提升技术水平,增进校内交流,提供必要的设施支持,搭建一个校内的,开放有趣的机器人
交流平台。随着团队的不断发展壮大,未来机器人团队逐渐积累了相当一批技术人才,并且开始有
机会参于国际比赛。其中Tinker机器人作为团队的主要开发项目,每年固定参加著名的RoboCup
比赛@HOME分赛场\cite{wisspeintner2009robocup}。

作为清华校内机器人方向最具影响力的学生社团,未来机器人团队在促进校内机器人相关领域的
社区发展,提高同学们的技术水平方面起到了重要影响。其中Tinker作为未来机器人团队的重要
支撑项目,在凝聚团队,提升技术水平,扩大队员眼界方面起到了重要的作用。并且Tinker本身
的技术迭代和技术积累也对校内科创发展起到了极大的推动作用,这些珍贵的技术经验不仅惠及
了未来机器人团队内部成员,也广泛的影响了校内科创领域的许多组织。

\begin{figure}
  \centering
  \includegraphics[width=.4\linewidth]{tinker_overview.jpg}
  \caption{Tinker外观}
  \label{fig:tinker_overview}
\end{figure}


\section{RoboCup@Home简介}

RoboCup@Home作为机器人领域的国际顶级赛事,旨在帮助提升机器人在复杂的开放环境中的表现,
促使家庭服务型机器人行业的发展。RoboCup@Home赛场分为三个联盟:Open Platform League
(OPL),Domestic Standard Platform (DSPL),Socal Standard Platform League
 (SSPL)。

三大赛事联盟中DSPL与SSPL均指定机器人的比赛,开发者不允许使用除特定机器人之外的任何机器人
参赛,且不允许替换、改装、外挂任何除机器人本身带有的部件,这两个联赛旨在促进社区内对相关
算法及软件工程的提升。其中DSPL规定使用Toyota Human Support Robot (HSR)\cite{toyota_hsr}
机器人,SSPL规定使用Softbank Robotics Pepper\cite{pandey2018mass}机器人。两款
机器人如图\ref{fig:hsr_pepper}所示。


OPL为开放平台比赛,参赛者必须自行设计搭造参赛机器人,并且为机器人编写特定的软件完成比赛
中制定的任务。OPL旨在通过开放的硬件要求培养机器人领域结构、硬件相关的人才,提升各个高校
和组织中的硬件设计能力,并探索结构、硬件方面的更多可能性。Robocup@Home自2009年开始举办,
多年的比赛中OPL赛场中涌现了大量的优秀机器人设计作品,Tinker即为其中优秀的一员。除Tinker
外,还有homer@ UniKoblenz\cite{memmesheimer2017homer},Pumas\cite{savage2013pumas},
CATIE Robotics\cite{fabre2018catie}等一批外观独特,功能强大的参赛作品出现,如图
\ref{fig:other_teams}。

Robocup@Home比赛每年举行一次,举行地点在国际几大城市巡回举办,由高校或者相关组织承办。
通常比赛分为3个stage,每个stage设有若干任务,参赛队伍一次完成任务,根据完成程度和完成时
的表现(速度,流畅性等等)按点得分,分数靠前的队伍可以获得晋级资格,进入下一stage。stage I
为基础功能比赛,主要考察避障导航、抓取、识别、语音交互等基础功能,辅以合理的机器人软件控制
程序即可完成,这一个stage主要考察机器人基本的软硬件设计的自恰性及稳定性,相当多新手队伍会
在stage I直接淘汰。

stage II的任务内容同stage I基本相当,但是难度会更高,逻辑也更复杂,特别的,在近几年的比赛中,
增加了General Purpose Service Robot(GPSR)任务,即裁判对机器人下达任意在Robocup@Home
赛场上曾经出现过的任务,考察机器人的完成情况,这一任务旨在帮助参赛者更好的构建机器人控制
软件,为家用机器人在真实场景下的应用做铺垫。

Final stage为两个得分最高的团队关于冠军奖项的争夺,每年的任务并不固定,2020年组委会
将送餐机器人作为Final Stage的考察目标,要求参赛队伍在一个嘈杂的餐厅环境中将食物托盘
送到某一个特定食客手上,并完成必要的交互。

\begin{figure}
\centering
\begin{subfigure}{.5\textwidth}
  \centering
  \includegraphics[width=.7\linewidth]{homer.png}
  \caption{homer@ UniKoblenz}
  \label{fig:homer}
\end{subfigure}%
\begin{subfigure}{.5\textwidth}
  \centering
  \includegraphics[width=.7\linewidth]{pumas.jpg}
  \caption{Pumas}
  \label{fig:pumas}
\end{subfigure}
\begin{subfigure}{.5\textwidth}
  \centering
  \includegraphics[width=.7\linewidth]{catie.jpg}
  \caption{CATIE Robotics}
  \label{fig:catie}
\end{subfigure}
\caption{RoboCup@Home指定机器人,左HSR 右Pepper}
\label{fig:other_teams}
\end{figure}

\section{RoboCup@Home比赛规则介绍}

经过多年的发展RoboCup@Home已经培育起一个完善的社区,每年RoboCup@Home的比赛任务
即由社区相关人员协助商议决定,RoboCup@Home组委会共同维护一份RuleBook\cite{rulebook}
,组委会成员
使用GitHub完成RuleBook的编写任务,该仓库是完全开放且欢迎参赛者提交修改与贡献的。
RoboCup@Home组委会中,设有专门的任务制定部门Technical Committee(TC),这部分成员
大都有深厚的相关从业背景和多年的研究经验,另外有一部分成员来自各个主力参赛队伍的
核心开发人员(比如笔者作为Tinker机器人的主要程序员参与了2020年的TC工作),以
确保比赛规则同时兼顾学科研究热点和可完成性。

随着机器人行业相关技术的不断发展,RuleBook中的任务也在每年更新,并且向着越来越复杂、
越来越贴近真实生活场景的方向演进。在多年的比赛中,组委会在任务内容、任务形式上也作出
了很多改进与尝试,2019年的RoboCup@Home比赛就及其大胆的将任务数量进行了爆炸式的扩容,
尝试了很多新的复杂的任务,其中有一些被证明并不成功,因此2020年的规则制作过程中又将
不合理的任务去除,将任务数量固定为Stage I 5个,Stage II 4个。

\subsection{RoboCup@Home经典任务分析}




\section{Tinker的历史版本}

在近5年的参赛过程中,Tinker机器人经过机器人队几代同学的不断开发,经历了若干次迭代,
其硬件设计、软件架构、传感器使用、方案选择都发生了本质的变化。在Tinker不断迭代的
过程中,机器人队也积累了大量的相关经验和教训,培养出了一代代机器人相关的人才。

早期机器人团队技术实力差资金紧张,因此我们自制了整个机器人参加比赛。当时的机器人如图
\ref{fig:old_tinker}所示。当时的虽然工艺粗糙,精度很低,但是辅以合适的
软件算法和恰当的策略,仍然在RoboCup的赛场上取得了不错的成绩。当时的队员们如今有的
已经走上工作岗位,有的还留在学校继续攻读更高的学位,都在各自的领域中继续发挥自己的
能量。


\begin{figure}[H]
\centering
\begin{subfigure}{.5\textwidth}
  \centering
  \includegraphics[width=.6\linewidth]{old_tinker1.jpg}
  \caption{早期Tinker侧视图}
\end{subfigure}%
\begin{subfigure}{.5\textwidth}
  \centering
  \includegraphics[width=.6\linewidth]{old_tinker2.jpg}
  \caption{早期Tinker主视图}
\end{subfigure}
\begin{subfigure}{.8\textwidth}
  \centering
  \includegraphics[width=.7\linewidth]{old_tinker3.jpg}
  \caption{早期Tinker在赛场上}
\end{subfigure}
\caption{RoboCup@Home指定机器人,左HSR 右Pepper}
\label{fig:old_tinker}
\end{figure}

经过几年的积累,机器人团队于2019年对Tinker进行了彻底的改装,包括
使用市售成品机械臂及夹爪、使用更高性能的笔记本个更高性能的电机等等
改造,外观如图\ref{fig:tinker1}。硬件的大幅改进对机器人开发带来
了极大的冲击,新版Tinker最初
的很多设计并不合适,因此经过2019年RoboCup@Home比赛后,我们又对Tinker
进行了一次大规模的改装。第二次改装吸取了前面的教训,大幅裁剪了很多
冗余设计,改良了供电和控制电路方案。在软件上也有诸多变化,一方面
固化了许多经过前人验证的优良开发范式,一方面积极缩减代码依赖,在
轻量化和稳定性方面不断提高,以争取更好的赛场表现。改装后的Tinker外观
如图\ref{fig:tinker2}所示。

\begin{figure}
\centering
\begin{subfigure}{.5\textwidth}
  \centering
  \includegraphics[width=.6\linewidth]{tinker1.png}
  \caption{改装Tinker 1.0}
  \label{fig:tinker1}
\end{subfigure}%
\begin{subfigure}{.5\textwidth}
  \centering
  \includegraphics[width=.6\linewidth]{tinker2.jpg}
  \caption{改装Tinker 2.0}
  \label{fig:tinker2}
\end{subfigure}
\caption{新版Tinker二次改装前后的外观,左改装前,右改装后}
\label{fig:tinker_new}
\end{figure}














