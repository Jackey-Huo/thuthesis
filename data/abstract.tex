% !TeX root = ../main.tex

% 中英文摘要和关键字

\begin{abstract}

  随着计算机技术的飞速发展以及智能制造水平的不断提高,智能机器人成为新的学科热点。其中家用智
  能机器人涉及领域众多,结构复杂,软硬件设计难度极大,且融合了机器人及相关行业的多项技术,
  具有极大的研究价值。
  
  本文介绍了清华大学未来机器人团队设计实现的多功能家用探索型机器人Tinker。本文以Tinker
  的设计与功能为核心,系统的介绍了智能机器人领域现阶段广泛使用的各种软硬件解决方案,
  并分析了在过去的几年时间中未来机器人团队在开发过程中积累的经验与教训。
  
  本文还详细的给出了Tinker机器人实现的各项功能,详尽的给出了各个功能实现过程中的
  演进过程、最终方案、性能开销和大致的软硬件成本,并且辅以全面的图片、视频资料。
  
  % 关键词用“英文逗号”分隔
  \thusetup{
    keywords = {Tinker, Robotics Learning, Domestic service robotics},
  }
\end{abstract}

\begin{abstract*}

  Intelligent robot in domestic environment is regard of utmost importance in the field of
  industrial and every-day life. Therefore many efforts have recently focused on
  this problem.

  In this paper, we describe the joint effort of the Team of Tinker in the past
  year. RoboCup@HOME consists of a settled set of benchmarking tests that cover multiple
  skills needed by domestic service robots. We present the hardware chosen, the approaches
  used and the system established to accomplish the tasks assigned in the paper, and also
  improvements achieved since RoboCup@HOME 2019. It includes a framework for behavior
  modeling and communication employed between robot and humans, as well as the policy
  decisions made within the robot itself. We describe our main contributions in arranging
  the logical model, integrated with various open source algorithm.

  \thusetup{
    keywords* = {Tinker, Robotics Learning, Domestic service robotics},
  }
\end{abstract*}
