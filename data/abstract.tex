% !TeX root = ../main.tex

% 中英文摘要和关键字

\begin{abstract}

  近些年随着机器人技术发展和应用场景的不断增多,移动型灵巧操作机器人逐渐成为了当前的研究
  热点。它不仅具备了灵活的移动能力,而且可以完成灵巧操作任务。本文开发了一款Tinker移动操
  作机器人系统,由7-DoF机械臂、二指自适应夹爪和麦克纳姆轮全向底盘组成。本文研究了机器人即时定位
  与建图技术,以Cartographer为基础搭建了自动导航避障的算法框架,开发了基于ROS的机器人
  路径规划与控制的软件模块。研究了机械手臂的操作控制方法。结合张正友视觉标定算法实现了相机
  的参数标定,并提出了一种快捷有效的自动化手眼标定方法。同时利用物体识别定位和抓起点识别算
  法,搭建完成了机器人灵巧操作的软件系统。最后在实际系统验证了方法的有效性。
  
  % 关键词用“英文逗号”分隔
  \thusetup{
    keywords = {Tinker, Robotics Learning, Domestic service robotics,
                Mobile Manipulation Robot, Mobile Manipulator},
  }
\end{abstract}

\begin{abstract*}

  In recent years, with the development of robot technology and the continuous
  increase of application scenarios, mobile manipulation robots have gradually
  become the current research hotspots. It has flexible mobility, and can also
  perform smart manipulation tasks. In this paper, we propose a new mobile
  manipulation robot: Tinker. As a robot system, it is composed of a 7-DoF
  robotic arm, a two-finger adaptive gripper and an omnidirectional chassis of
  Mecanum wheels. This paper developed an software framework for automatic navigation
  and obstacle avoidance based on Cartographer, and a robot control
  system for path planning and arm control based on ROS was developed. This paper
  studied the manipulation control method of multiple freedom robot arm. In this
  paper, an automatic hand-eye calibration framework is present. Meanwhile, a
  software system of computer vision based on RGB camera and point cloud data
  combined with robot arm control and grasp point selection
  is designed. Finally, the effectiveness of the method is verified in the real
  world.

  \thusetup{
    keywords* = {Tinker, Robotics Learning, Domestic service robotics,
                 Mobile Manipulation Robot, Mobile Manipulator},
  }
\end{abstract*}
