% !TeX root = ../main.tex

\chapter{Tinker硬件设计介绍}
\label{cha:chapter02}


\section{Tinker机械结构设计}

按结构划分,可以将Tinker的机械结构粗略的划分为头部,机械臂,骨架,身体主控
以及底盘。机器人的整体外观如图\ref{fig:tinker_front_side}所示。

\begin{figure}[h] % use float package if you want it here
  \centering
  \includegraphics[width=1.\linewidth]{tinker_front_side.png}
  \caption{Tinker前视图与侧视图}
  \label{fig:tinker_front_side}
\end{figure}

\subsection{Tinker骨架搭建}
为方便交通运输与结构迭代,Tinker基本骨架使用40cm*40cm铝形材搭建。骨架
的主要作用为搭建起整个机器人的框架,将各个传感器假设到理想的位置上,并且
为机器人的主控、供电、走线留出足够的空间。经过多轮版本迭代后,目前Tinker
底盘尺寸固定在35cm×45cm,净高140cm,总质量70kg。

为了方便Tinker机器人机械结构长期的维护,降低人员交接成本,我们将Tinker
使用的型材尺寸做了规范化处理,并且对骨架进行了手动建模,以方便机器人的拆
装与改进,如图\ref{fig:base_link}所示。

\begin{figure}[h] % use float package if you want it here
  \centering
  \includegraphics[width=.4\linewidth]{base_link.png}
  \caption{Tinker骨架的STL建模结果}
  \label{fig:base_link}
\end{figure}

\subsection{Tinker头部设计及版本迭代}

新版的Tinker头部主要用来放置顶部摄像头,在背景环境特别复杂的情况下还会在头部
旁边加装专业的采音麦克风用于语音识别和声源定位。在一般情况下使用头部安装的kinect v2
摄像头自带的麦克风阵列即可满足需求。在Tinker中期我们曾经使用过2自由度云台支持
头部摄像头,使头部摄像头具有上下和左右转动的能力,以扩大摄像头的视野,如图\ref{fig:neck}
所示。该转台是机器人团队根据现有的通用二自由度转台自行改装的,主要改动有:更换了
原有的舵机,改为精度更高且有位置反馈的dynamixel(TODO:具体型号需要核实)舵机,
并且相应的改装了转台本身的尺寸和各种装配零件。客观上这一设计确实提高了机器人头部
传感器的视野,但是经过漫长的实践,这一结构被证明是一个彻头彻尾的失败设计。原因有
二:dynamixel本身需要合理的供电和信号线,摄像头本身也有相应的传输线,这些线在频繁
转动的时候很容易被拉断造成整个控制系统崩溃;摄像头本身作为抓取和部分避障的主力
传感器,其位置精度是非常重要的,但是由于舵机空程以及装配过程中产生的空程等等问题,
转台的位置精度一直无法达到要求,大大减弱了摄像头的数据效果。因此在新版的Tinker
中,转台这一设计被机器人团队彻底抛弃了。

\begin{figure}[h] % use float package if you want it here
  \centering
  \includegraphics[width=.4\linewidth]{neck.png}
  \caption{早期Tinker头部的2自由度转台}
  \label{fig:neck}
\end{figure}

\subsection{Tinker机械臂方案选择}

Tinker使用的机械臂为Universal Robotics公司生产的6自由度机械臂UR5。在决定使用
该机械臂之前我们对市面上在售的各种成品机械臂进行了广泛的调研,包括RoboCup@Home
赛场上使用过的Kinova\ref{fig:kinova}
机械臂配合三指夹爪,Retinker Robotics公司开发的Sawyer\ref{fig:sawyer}
机械臂,以及Universal Robotics公司出产的各个型号的机械臂\ref{fig:ur_series}。
最终综合考虑了价格、售后、供电、开发者社区等
各个因素后,我们最终确定使用了UR5这个型号的机械臂,并且对机械臂的控制与供电进行了
适当的改装,以符合比赛的需要。

\begin{figure}
\centering
\begin{subfigure}{.5\textwidth}
  \centering
  \includegraphics[width=.9\linewidth]{kinova.png}
  \caption{Kinova}
  \label{fig:kinova}
\end{subfigure}%
\begin{subfigure}{.5\textwidth}
  \centering
  \includegraphics[width=.5\linewidth]{sawyer.jpg}
  \caption{Sawyer}
  \label{fig:sawyer}
\end{subfigure}
\begin{subfigure}{.8\textwidth}
  \centering
  \includegraphics[width=.7\linewidth]{ur_series.jpg}
  \caption{UR系列机械臂}
  \label{fig:ur_series}
\end{subfigure}
\caption{机器人队前期调研时关注过得机械臂型号}
\label{fig:arms}
\end{figure}

早期夹爪使用Robotiq
公司生产的Robotiq 2f-85二指夹爪(最大张角85mm),后经过不断的测试与赛场上的较量,
我们发现2f-85的张角对于家庭机器人来说过窄,于是换用了robotiq 2f-140款式的二指夹爪,
该夹爪张角为140mm,能够更好的满足抓取物品、拾取袋子的需求。因此Tinker团队最终使用
UR5 + Robotiq 2f-140作为最终比赛方案,如图\ref{fig:ur5_2f140}所示。

\begin{figure}[h] % use float package if you want it here
  \centering
  \includegraphics[width=.4\linewidth]{ur5_2f140.jpeg}
  \caption{Tinker最终使用的机械臂及夹爪方案}
  \label{fig:ur5_2f140}
\end{figure}

\subsection{Tinker主控与供电}

Tinker的核心控制由一台搭载了i9-9900K cpu及RTX2080显卡的笔记本电脑完成。在前
几代Tinker设计中,一般主控由工控机或者经过电源改装的台式机担任,但是随着笔记本
性能的不断增强,我们最终选择使用成品机器完成这一任务。成品笔记本性能好,散热完善
且主板设计精良,稳定性好,并且笔记本有自己独立的供电系统,可在机器人断电时单独运行
大大的提升了系统的稳定性和易用性。

整个机器人的供电系统都放在机器人腹部,包括机械臂、底盘、各种传感器、笔记本电脑所需
的所有供电电路以及电池组。2019年Tinker使用单一磷酸铁锂电池供电,标准输出电压29.4V
到24V,容量50AH,放在底盘上部。经过19年比赛的反思之后,机器人队认为,使用单一锂电池
供电太过危险,且参赛过程中运输太过麻烦,成本过高,于是我们将供电方案改成了多块标准
29V 1500mAH电池并接供电的形式,并3D打印了特制的电池架将电池倒挂在底盘下,这样一方面
大大缩小了Tinker底盘的面积提高了底盘避障导航的灵活性,另一方面降低了整个机器人的中心,
使得机械臂运动时的型变更小,提高了机器人的性能。在供电电路布线方面,Tinker也有了
显著的改进,前一版本中Tinker直接借用了UR5自带的配电箱摆放机械臂及其他部件需要的
供电线路、DCDC、分流排等等元件。但是配电箱本身尺寸过大,不透明,且强度过大,给机器人
装配、调试带来了很大的困难,也对导航的表现造成了影响。之后我们在充分考虑了散热、消防
等等制约之后,使用亚克利版重新制作了一版配电箱,且对机器人的供电线路进行了梳理,大大
缩小了机器人的底盘尺寸,也使得整个供电设计更加简洁易于调试。前后两个版本的供电外观
如图所示。

\begin{figure}
\centering
\begin{subfigure}{.5\textwidth}
  \centering
  \includegraphics[width=.6\linewidth]{tinker_elec.jpg}
  \caption{改版前的电池及供电线路放置}
  \label{fig:kinova}
\end{subfigure}%
\begin{subfigure}{.5\textwidth}
  \centering
  \includegraphics[width=.7\linewidth]{tinker_elec_new.png}
  \caption{改版后的电池及供电线路放置}
  \label{fig:sawyer}
\end{subfigure}
\caption{Tinker前后两个版本的供电布局}
\label{fig:arms}
\end{figure}

\subsection{Tinker底盘设计}

Tinker的底盘设计受前几版机器人的影响最多,早期的Tinker使用自制的3自由
度机械臂,其精度有限且自由度受限太严重,需要具有左右额外自由度的底盘
来弥补这一不足,因此Tinker的底盘被设计为使用4个麦克纳母轮的万向底盘,底盘
设计如图所示。

TODO:插入麦轮的底盘的solidworks图。

新版Tinker改为成品机械臂后抓取的自由度和覆盖范围都有了质的飞跃,不再要求
底盘具有万向移动的性能了。但是经过多年的改进,机器人团队中对万向底盘的制作
技术已经成熟,于是我们继续沿用了这一设计,仅在电机选型上做了一些与时俱进
的改良,选用了扭矩较大且噪音较小的大疆 GM6020直流无刷电机做底盘的动力电机,
如图 \ref{cha:chapter02}所示。

\begin{figure}[h] % use float package if you want it here
  \centering
  \includegraphics[width=.7\linewidth]{chassis.png}
  \caption{Tinker底盘的电机传动结构}
  \label{fig:ur5_2f140}
\end{figure}

























