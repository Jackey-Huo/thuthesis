\thusetup{
  %******************************
  % 注意:
  %   1. 配置里面不要出现空行
  %   2. 不需要的配置信息可以删除
  %******************************
  %
  %=====
  % 秘级
  %=====
  secretlevel={秘密},
  secretyear={10},
  %
  %=========
  % 中文信息
  %=========
  ctitle={基于非监督学习的视觉里程计\\阶段研究报告},
  cdegree={工学硕士},
  cdepartment={计算机科学与技术系},
  cmajor={计算机科学与技术},
  cauthor={霍江浩},
  csupervisor={宋亦旭},
  ccosupervisor={郭岚}, % 联合指导老师
  % 日期自动使用当前时间,若需指定按如下方式修改:
  % cdate={超新星纪元},
  %
  % 博士后专有部分
  %catalognumber     = {分类号},  % 可以留空
  %udc               = {UDC},  % 可以留空
  %id                = {编号},  % 可以留空: id={},
  %cfirstdiscipline  = {计算机科学与技术},  % 流动站(一级学科)名称
  %cseconddiscipline = {深度学习},  % 专 业(二级学科)名称
  %postdoctordate    = {2019 年 7 月——2019 年 8 月},  % 工作完成日期
  %postdocstartdate  = {2019 年 7 月 10 日},  % 研究工作起始时间
  %postdocenddate    = {2019 年 8 月 31 日},  % 研究工作期满时间
  %
  %=========
  % 英文信息
  %=========
  etitle={Unsupervised Learning based Visual Odometry Research},
  % 这块比较复杂,需要分情况讨论:
  % 1. 学术型硕士
  %    edegree:必须为Master of Arts或Master of Science(注意大小写)
  %             “哲学、文学、历史学、法学、教育学、艺术学门类,公共管理学科
  %              填写Master of Arts,其它填写Master of Science”
  %    emajor:“获得一级学科授权的学科填写一级学科名称,其它填写二级学科名称”
  % 2. 专业型硕士
  %    edegree:“填写专业学位英文名称全称”
  %    emajor:“工程硕士填写工程领域,其它专业学位不填写此项”
  % 3. 学术型博士
  %    edegree:Doctor of Philosophy(注意大小写)
  %    emajor:“获得一级学科授权的学科填写一级学科名称,其它填写二级学科名称”
  % 4. 专业型博士
  %    edegree:“填写专业学位英文名称全称”
  %    emajor:不填写此项
  %edegree={Doctor of Engineering},
  %emajor={Computer Science and Technology},
  %eauthor={Xue Ruini},
  %esupervisor={Professor Zheng Weimin},
  %eassosupervisor={Chen Wenguang},
  % 日期自动生成,若需指定按如下方式修改:
  % edate={December, 2005},
  %
  % 关键词用“英文逗号”分割
  ckeywords={SLAM, 非监督学习, 深度学习, 视觉里程计},
  %ekeywords={TeX, LaTeX, CJK, template, thesis}
}

% 定义中英文摘要和关键字
\begin{cabstract}

  即时定位建图技术(SLAM)通过特定的传感器完成实时的定位、建图任务,在机器人研究
  领域一直有着重要的地位,是机器人完成复杂任务的基石,视觉里程计(VO)作为SLAM的重要分支
  在机器人领域也占有着重要的地位。传统VO算法主要依赖激光雷达或者单目双目摄像头,基于滤波
  或者优化算法完成定位任务。
  
  近些年来,随着深度学习的兴起,SLAM领域的研究前沿开始逐渐使用深度学习方法进行探索,
  同传统算法相比,深度学习方法鲁棒性强、配置简单、代码量小并且易于理解。本文根据领域
  内已有的成果,进一步探索了非监督学习的视觉VO技术,复现了一种基于单目且有绝对尺度
  的非监督VO网络,具体的分析了该网络的各项技术设计,并且给出了阶段性的实验成果。

\end{cabstract}

% 如果习惯关键字跟在摘要文字后面,可以用直接命令来设置,如下:
% \ckeywords{\TeX, \LaTeX, CJK, 模板, 论文}

\begin{eabstract}
   An abstract of a dissertation is a summary and extraction of research work
   and contributions. Included in an abstract should be description of research
   topic and research objective, brief introduction to methodology and research
   process, and summarization of conclusion and contributions of the
   research. An abstract should be characterized by independence and clarity and
   carry identical information with the dissertation. It should be such that the
   general idea and major contributions of the dissertation are conveyed without
   reading the dissertation.

   An abstract should be concise and to the point. It is a misunderstanding to
   make an abstract an outline of the dissertation and words ``the first
   chapter'', ``the second chapter'' and the like should be avoided in the
   abstract.

   Key words are terms used in a dissertation for indexing, reflecting core
   information of the dissertation. An abstract may contain a maximum of 5 key
   words, with semi-colons used in between to separate one another.
\end{eabstract}

% \ekeywords{\TeX, \LaTeX, CJK, template, thesis}
